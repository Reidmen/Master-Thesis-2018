\begin{intro}

En el contexto de comportamiento mecánico en medicina, hay abundantes preguntas a ser respondidas fluctuando desde el modelamiento mismo de distintos organis, procedimientos experimentales junto con simulaciones, a la verificación y predicción de diferentes factores de interés como son la densidad, la rigidez, etc. Este trabajo está orientado en esta dirección, el cual mediante técnicas de la teoría de homogenización es estudiado el hueso mediante modelamiento multi-escala usando su composición y geometría de microestructura caracteristica.

Desde un punto experimental es modelado el procedimiento experimental de ultrasonido, obteniendose resultados fidedignos para el comportamiento del hueso comparable con datos experimentales reales. 
Desde un punto teórico, se justifica mediante teoría espectral y aproximación asintótica de dos escalas la existencia de los problemas multiscala como las ecuaciones effectivas (macroscopicas) usadas para obtener soluciones numericas eficientes mediante la libreria \texttt{FEniCS}.
\\

In the context of mechanical behavior in medicine, there are abundant question to be answer, fluctuating from the mathematical modelling itself of different organs and physiological structures, experimental procedures of clinical relevance and their simulations to the verification and prediction of different factor of clinical interest such as density, stiffness constants, etc. 
This work is oriented mainly in the modelling and numerical implementation of a experimentally proposed procedure, where using homogenization theory techniques it is studied the bone material mechanical behavior by means of a multiscale framework in a model that incorporates the characteristic microstructure associated to it, so to called mesoscale.
Over such a model its derived the so-called cell problems that incorporated the non-linearity of the two-scale asymptotic approximation being used.

Such a workflow definition let us then obtain the macroscopic equations governing the model that incorporated the intrinsic non-linear microstructure behavior and gives us explicit expression for the model and an algorithm that predict the macroscopic (or effective) model underlying the initial setting, allowing us an explicit formulation and then implementation of it using an state-of-art library called \texttt{FEniCS} to compute the solutions of such kind of systems.

Modelling the ultrasound experimental procedure proposed from \textit{Minonzio, Foiret} \cite{Foiret2014}, its then obtained the numerically curves of Lamb waves and singular values ressembling the real data. In a similar fashion, the solution of the homogenized coefficients that model the effective mechanical behavior are comparable with the reference literature, providing a standpoint for future studies.

Moreover, from a theoretical stand-point its justified the two-scale asymptotic expansion being used by means of a convergence result of the solution associated to the mathematical model proposed for the experimental procedure that takes into account a mixed boundary condition problem.

In a similar fashion, its studied variations to the model problems to account effects of resonance behavior in the frequency domain, and instabilities oberved in the spectral decomposition of the operator. 
Also, it's studied the effects of viscoelasticity behavior in the model using relevant literature, searching for relevant viscous effects shown in literature and modelled by a small set of parameters modelled in an adequate mechanical behavior.


\end{intro}