\chapter{Discretization Procedures}



\section{The Elastodynamic Model}
\subsection{Approximation by Finite Spaces}
To discretize the elastodynamic model, we consider the elastic operator defined on a \textit{Banach} space $\mathbf{V}$ with elasticity tensor $\mathbf{C} = (C_{ijkl})_{ijkl}$ by:
\begin{equation*}
    \mathcal{I}_C (u,v) := \int \limits_{\Omega} C_{ijkl}\mathbf{e}_{kl}(u) \mathbf{e}_{ij}(v) \, dx \quad \forall u,v \in \mathbf{V}
\end{equation*}
Moreover, by the uniform ellipticity condition on the coefficients $C_{ijkl}$ we have that $\mathcal{I}_{C}(\cdot, \cdot)$ is coercive on $\mathbf{V}$, i.e., there exist $\alpha > 0$ constant, such that:
\begin{equation*}
    \mathcal{I}_C (u,u) \geq  \alpha \Vert u \Vert_{\mathbf{V}}
\end{equation*}

The solution of the elastodynamic problem will be found as projection solutions of finite dimensional subspaces that approximate to solution space \footnote{Such workflow is defines the approximation theory of PDE solutions and the so-called finite elements method \cite{ern2004theory}.}, in this case $\mathbf{H}^1(\Omega, \Gamma_D)$.
Let $\{\mathbf{V}_h \}_{0 \leq h \leq 1}$ be a family of finite-dimensional subspaces of $\mathbf{V}$, we assume that there exist a dense subspace $\mathbf{W} \subset \mathbf{V}$, a linear interpolator $\Pi_h \in \mathcal{L}(\mathbf{W};\mathbf{V}_h)$, integer $k > 0$ such that our finite dimensional solution approximation $\Pi_{h} $ with associated to a mesh size $h>0$ satisfies:
\begin{equation*}
    \Vert v - \Pi_h v \Vert_{\mathcal{L}(\mathbf{W};\mathbf{V}_h)} \leq h^{k+1} \Vert v \Vert_{\mathbf{W}} \quad \forall v \in \mathbf{W}
\end{equation*}
%In particular, the above assumptions are assured in REFERENCE: HUDGENDS PAPER.
\begin{rem}
In our case under study its used $\mathbf{V} = \mathbb{L}^2(\Omega)$ and $\mathbf{W} = \mathbf{H}^1(\Omega, \Gamma_D)$ being $\mathbf{V}_h$ the $\mathbf{H}^1$-conformal finite element spaces using as reference element $\{ \hat{K}, \hat{P}, \hat{\sigma} \}$ of \textit{Lagrange} type with typical order $k \in \{1,2\}$, such that $\mathbb{P}_k \subset \hat{P}$ with $\mathbf{W} := \mathbf{H}^{k+1}(\Omega) \cap \mathbf{H}^1(\Omega, \Gamma_D)$
\end{rem}

Over such finite dimensional subspaces $\{ \mathbf{V}_h\}_{h>0}$ the approximate solution can be found, satisfying the problems:
\begin{equation}
\label{ODE-Discretized}
    \left \{
    \begin{array}{cc}
        \text{Find } u_h \in \mathcal{C}^2(0,T; \mathbf{V}_h) & \text{ such that } \\
        (d_{tt} u_h(t), v_h)_{\Omega} + \mathcal{I}_C(u_h(t), v_h) = (\mathbf{F}(t) v_h)_{\Gamma_N} & \forall v_h \in \mathbf{V}_h\\
        d_t u_h(0) = u_h(0) = \mathbf{0} & 
    \end{array}
    \right. 
\end{equation}

Problems (\ref{ODE-Discretized}) corresponds to a finite linear PDE system in time for the variable $u_h(t)$ where the existence can be obtained by application of the spectral theory decomposing the elastic operator in its spectrum. The treatment is similar to the procedure proposed in the above section \cite{raviart1983introduction}.

The space discretization then is obtained from (\ref{ODE-Discretized}) which corresponds to a system of ODE's. Indeed, let $\{ \varphi_1, \dots, \varphi_{N(h)} \}$ denote a basis for the finite dimensional space $\mathbf{V}_h$ or so-called global shape functions for $\mathbf{V}_h$, then for all $t \in [0,T]$ the approximate solution $u_h(t) \in \mathbf{V}_h$ can be expanded as
\begin{equation*}
    u_h(\mathbf{x},t) = \sum \limits_{l=1}^{N(h)} \mathbf{U}_i(t) \varphi_i
\end{equation*}
where for each $t \in [0,T]$ its used the notation $\mathbf{U}(t) = (U_i(t))_{1 \leq i \leq N} \in \mathbb{R}^N$.

Now, introducing also $\mathbf{F}_h(t) = \big( (\mathbf{F}(t), \varphi_i)_{\Omega} \big)_{1 \leq i \leq N} \big) \in \mathbb{R}^N$ with the so-called stiffness $\mathcal{A}(t) \in \mathbb{R}^{N \times N}$ and stress $\mathcal{M} \in \mathbb{R}^{N \times N}$ matrices respectively by:
\begin{equation*}
    \mathcal{A}_{ij}(t) = \mathcal{I}_C(\varphi_j, \varphi_i), \, \mathcal{M}_{ij} = (\sqrt{\rho^0} \varphi_i , \sqrt{\rho^0}\varphi_j)_{\Omega} \quad 1 \leq i,j \leq N
\end{equation*}

By the previous assumed hypothesis $\mathcal{A}(t)$ is symmetric and positive-definite whereas $\mathcal{M}_{ij}$ is positive definite, Under such notation, it follows the matrix form formulation time-dependent:
\begin{equation}
    \label{MatrixFormulation}
    \left \{
    \begin{array}{cc}
        \text{Find } \mathbf{U} \in \mathcal{C}^2(0,T; \mathbb{R}^{N\times N}) & \text{ such that} \\
        \mathcal{M} d_{tt} \mathbf{U}(t) + \mathcal{A} \\ \mathbf{U}(t) = \mathbf{F}_h (t) & \text{ in }(0,T) \\
        d_{t} \mathbf{U}(0) = \mathbf{U}(0)  = \mathbf{0}&
    \end{array}
    \right.
\end{equation}
which describes to complete space discretization by means of the FEM method, and enabling the use of time discretization techniques for the resulting ODE problem (\ref{MatrixFormulation}). In particular as will be seen in the next section the so-called $\beta$-Newmark scheme. \footnote{The implementation of such space discretization is done in Python using a state-of-art open-source project called \texttt{FEniCS}, defined as a set of specific core component such as \texttt{DOLFIN}, \texttt{UFL},  \texttt{UFC}, etc. It enables the automatic solution of differential forms implementing the Theory of Finite Elements with numerically optimized libraries for arithmetic computations. The project is an international collaboration initiated in 2003 between the University of Chicago and Chalmers University of Technology currently under development from a variety of institutions.}

\subsection{Time Discretization}
Let us note first that (\ref{MatrixFormulation}) is an standard \textit{Cauchy} ODE problem, To solve numerically such system, we will introduce an uniform time-step in the form $\Delta t = T/N_T$ with a discretization of the interval $[0, T]$ in the form:
\begin{equation*}
    t_n = n \Delta t, \quad 0 \leq n \leq N_T
\end{equation*}
being $N_T \in \mathbb{N}^*$ some number of steps.

We will find for each $n \in \{1,\dots, N_T \}$ an approximate sequence tuple $(u_n, v_n, a_n)$ of the pair $(u(t_n), \partial_{t} u(t_n), \partial_{tt} u(t_n))$ being the method chosen the so-called $\beta$-\textit{Newmark}, a well-known scheme to study mechanically driven behavior \cite{raviart1983introduction}.

Taking into account the time-domain regularity from the solution of the homogenized PDE problem (results analogous to \ref{BootstrapingProp}) let us note that by using the \textit{Taylor} expansion on the function $u \in \mathcal{C}^{p}(0,T;\mathbf{H}^1(\Omega, \Gamma_D))$ and the mean value theorem that for the displacement at time $t_{n+1}$ we have:
\begin{equation*}
    u(t_{n+1}) = u(t_n) + \Delta t \, \partial_{t} u(t_n) + \Delta t^2 \, \big( \beta \partial_{tt} u(t_{n+1} + (\frac{1}{2} - \beta) \partial_{tt} u(t_{n+1}) \big) + \mathcal{O}(\Delta t^3)    
\end{equation*}
and for the velocity
\begin{equation*}
    \partial_{t} u(t_{n+1}) = \partial_{t} u (t_{n}) + \Delta t \, \big( \gamma \partial_{tt} u(t_{n+1}) + (1-\gamma) \partial_{tt} u(t_n) \big) + \mathcal{O}(\Delta t^2)
\end{equation*}
being $\beta, \gamma$ two parameters with $0 < 2\beta, \gamma < 1$.
The $\beta$-Newmark scheme consist then in a approximation of the above equations by using a finite difference scheme in the form:
\begin{align}
    \label{TimeDif-Scheme}
    u_{n+1} &= u_{n} + \Delta t\, v_{n} + \Delta t^2 \, \big( \beta a_{n+1} + (\frac{1}{2} - \beta) a_n \big) \\
    v_{n+1} &= v_n + \Delta t\, \big( \gamma a_{n+1} + (1-\gamma) a_{n} \big)
\end{align}

From (\ref{TimeDif-Scheme}) its defined the numerical procedure by a two-step update given as follows:

\begin{align}
    \label{TwoStage-Update}
    \text{(Stage 1)}\longrightarrow  a_{n+1} &= \frac{1}{\beta \Delta t^2} ( u_{n+1}-u_n - \Delta t \, v_n) - \frac{(1-2\beta)}{2 \beta} a_n, \quad \forall \, 0 \leq n \leq N_T-1 \\
    \text{(Stage 2)}\longrightarrow v_{n+1} &= v_n + \Delta t \, \big( (1-\gamma) a_n + \gamma a_{n+1} \big), \quad \forall \, 0 \leq n \leq N_T-1
\end{align}

Let us now show the precision of the $\beta$-\textit{Newmark} scheme. Note that the elastodynamic equation contains two derivatives in time, so that taking into account the above and applying again \textit{Taylor} expansion and the mean value theorem, it follows as $\Delta t$ tends to zero that:

\begin{multline}
    u(t_{n+1}) = u(t_{n}) + \Delta t \, \partial_{t} u(t_n) + \\
    \Delta t \, \big( \beta \partial_{tt} u(t_{n+1}) + (\frac{1}{2}- \beta) \partial_{tt} u(t_n) \big) + \Delta t^3 (\frac{1}{6}-\beta) \partial_{ttt}u(t_n) + \mathcal{O}(\Delta t^4)
\end{multline}

\begin{multline}
    \partial_{t} u(t_{n+1}) = \partial_{t} u(t_n) + \Delta t\, \big( \gamma \partial_{tt} u(t_{n+1}) + \\
    (1-\gamma) \partial_{tt}u(t_n) \big) + \Delta t^2 \, (\frac{1}{2}-\gamma) \partial_{ttt} u(t_n) + \mathcal{O}(\Delta t^3)
\end{multline}

so that, the scheme results of precision $\mathcal{O}(\Delta t)$ for $\gamma \neq 1/2$ and $\mathcal{O}(\Delta t^2)$ for $\gamma = 1/2$.

Applying now such two-step scheme (\ref{TwoStage-Update}) over (\ref{MatrixFormulation}) it follows the matrix system update procedure:
\begin{align}
    \label{MatrixSystemUpdate}
    \frac{1}{\beta \Delta t} \mathcal{M}(\mathbf{U}_{n+1} - \mathbf{U}_{n} - \Delta t\, \mathbf{V}_n) - \frac{1}{\beta}(\frac{1}{2}- \beta) \mathcal{M}\mathbf{A}_n + \mathcal{A}\mathbf{U}_{n+1} &= \mathbf{F}_h \quad 0 \leq n \leq N-1 \\
    \frac{1}{\gamma \Delta t} \mathcal{M}(\mathbf{V}_{n+1} - \mathbf{V}_{n}) - \frac{1}{\gamma}(1-\gamma) \mathcal{M}\mathbf{A}_n + \mathcal{A}\mathbf{U}_{n+1} & = \mathbf{F}_h \quad 0 \leq n \leq N-1
\end{align}

Note then that such a time step iteration (\ref{MatrixSystemUpdate}) corresponds to solve the linear system of equations defined for the function $\mathbf{U}_{n+1}$ in the form:
\begin{equation*}
    (\mathcal{M}+ \beta \Delta t^2 \, \mathcal{A}) \mathbf{U}_{n+1} = \mathbf{F}_{n+1}
\end{equation*}
being $\mathbf{F}_{n+1} \in \mathbb{R}^{I(h)}$ known and then update the solution $\mathbf{U}_n$ field using the (\ref{TwoStage-Update}).
Now, since the $\beta \geq 0$, if follows that the $\mathbf{R}^{I(h)} \times \mathbf{R}^{I(h)}$ matrix $(\mathcal{M}+ \beta \Delta t^2 \mathcal{A})$ is positive definite and moreover its symmetric. 


\section{Fully Elastodynamic Model}
Taking into account our homogenized elastodynamic model (\ref{HomogenizedPDE}) described explicitly:
\begin{equation}
    \label{HomPDE-Model}
    \left \{
    \begin{array}{cc}
        \text{Find $u \in \mathcal{C}^2(0,T;\mathbf{H}^1(\Omega, \Gamma_D))$} & \text{such that} \\
        \rho^{hom} \partial_{tt} v(\mathbf{x}) (\mathbf{x}) - \nabla \cdot \sigma^{hom} (u^{(0)}(\mathbf{x}) ) = \mathbf{0} & \text{ in } \Omega \\
        \sigma^{hom}_{ij}(u^{(0)}(\mathbf{x})) = C^{hom}_{ijkl}\mathbf{e}_{kl,x}(\hat{v}(\mathbf{x})) & \text{ in } \Omega \\
        u^{(0)}(\mathbf{x}) = \mathbf{0} & \text{ on } \Gamma_D \\
        \sigma^{hom}(u^{(0})(\mathbf{x})) \cdot n = \mathbf{F}(\mathbf{x}) & \text{ on } \Gamma_N
    \end{array}
    \right .
\end{equation}
the discretization procedure over time applied on (\ref{HomPDE-Model}) turn into the update scheme for each $n \in \{1,\dots, N\}$ in the form:
\begin{equation*}
    \frac{\rho^{hom}}{\beta (\Delta t)^2} u_{n+1} - \nabla \cdot \sigma(u_{n+1}) = \frac{\langle\rho\rangle}{\beta (\Delta t)^2} ( u_{n} + (\Delta t) v_n ) + \langle\rho\rangle\frac{(1-2\beta)}{2\beta} a_n
\end{equation*}
with fields update (\ref{TwoStage-Update}).

\section{Elastodynamic Model with Attenuation}
The presence of eigen-frequencies in the elastodynamic operator gives a bad numerical solution after solving the discrete system in the frequency domain. This problem can be bypassed using a regularization factor, or moreover, a kind of \textit{viscosity} term which changes the eigen-frequencies and stabilizes the discrete system resulting in \textit{good} approximation to the experimental results.\\
In this case, we add a term which in \textit{viscoelastic} literature ressembles a \textit{Kelvin-Voigt} type beahvior. We add such term scaled with parameter $\epsilon$, so that the overall behavior is not abruptly disrupted but is well-posed in the set of real frequencies.

\begin{rem}
One possible way to see this technique is to consider the fact that we are moving the real frequencies $\omega$ to $\omega - i\epsilon$, so we are away from the eigen-frequencies of the operator $\mathcal{I}_{C^{hom}}$, but the overall behavior of the real part remains close to the experimental setting.
\end{rem}

So, we model the material as function of displacement $u(\mathbf{x},t) \in H^{1}(\Omega)$ (after the homogenization procedure) being $\Omega \subset \mathbb{R}^2$ a sufficiently regular domain, with Lipschitz boundary $\partial \Omega := \Gamma_D \cup \Gamma_N$. 
We model regularized the elastic model presented before in the form:
\begin{equation*}
    \left \{
    \begin{array}{cc}
        \rho^{hom} \partial_{tt} u^{(0)} - \epsilon \partial_{t} u^{(0)} - \nabla \cdot \sigma^{hom}(u^{(0)}) = \mathbf{0} & \text{in } \Omega \times (0,T) \\
        \sigma^{hom}(u^{(0)}(\mathbf{x},t)) =  \mathbf{C}^{hom}:\mathbf{e}(u^{(0)}(\mathbf{x},t))  & \text{in }\Omega \times (0,T)\\ 
        \sigma^{hom}(u^{(0)})\cdot n = \mathbf{F} & \text{on }\Gamma_N\times (0,T) \\
        u^{(0)} = \mathbf{0} & \text{on }\Gamma_D \times (0,T) \\
        u^{(0)} = \partial_t u^{(0)} = \mathbf{0}& \text{ on } \Omega \times \{t=0\}
    \end{array}
    \right .
\end{equation*}
where $\epsilon$ denotes a small regularization parameter associated to a Kelvin-Voigt type viscosity. 
\subsection{Time-discretization}
As before, we consider a time-discretization following the $\beta$-Newmark scheme following the two-step update procedure for each $n \in \{1,\dots, N\}$:
\begin{align*}
    \text{(Stage 1)} &\longrightarrow a_{n+1} = \frac{1}{\beta (\Delta t)^2} (u_{n+1}-u_{n}-(\Delta t)v_n) - \frac{1-2\beta}{2\beta}a_n\\
    \text{(Stage 2)}& \longrightarrow v_{n+1} = v_n + \Delta t((1-\gamma)a_n + \gamma a_{n+1})
\end{align*}
It's then possible to define a numerically explicit scheme for our model, with update procedure at time $n+1$ given in the form:
\begin{align*}
    \frac{(\langle \rho \rangle- \epsilon (\Delta t) \gamma)}{\beta (\Delta t)^2} u_{n+1} - \nabla \cdot u_{n+1} &= \frac{\langle \rho \rangle}{\beta (\Delta t)^2} (u_n + (\Delta t)v_n) + \langle \rho \rangle\frac{ (1-2\beta)}{2\beta}a_n \\
    & \, - \frac{\epsilon (\Delta t)\gamma}{\beta (\Delta t)^2}(u_n + (\Delta t)v_n) - \epsilon (\Delta t)\gamma\frac{1-2\beta}{2 \beta} a_n
\end{align*}

\section{A Kelvin-Voigt Viscoelastic Model}
Following the developments of the above ideas, a $\epsilon$-Kelvin-Voigt viscoelastic bone model captures the main characteristics observed in biological and biomechanical literature of bone. In this case, the model we consider is a linear elastic-viscous characteristic behavior with an $\epsilon$-regularization of the viscous part.
Then the multiscale system we are considering is defined after homogenization by
\begin{equation*}
    \left \{
    \begin{array}{cc}
        \rho^{hom} \partial_{tt} u - \nabla \cdot \sigma^{hom}(u^{(0)}) = 0 & \text{ in } \Omega \times (0,T) \\
        \sigma^{hom}(u^{(0)}(\mathbf{x},t)) =  \mathbf{C}^{hom}:\mathbf{e}(u^{(0)}) + \epsilon \mathbf{D}^{hom}:\mathbf{e}(\partial_{t}u^{(0)}) & \text{ in }\Omega \times (0,T)\\
        \sigma^{hom}(u^{(0)})\cdot n = \mathbf{F} & \text{ on }\Gamma_N\times (0,T) \\
        u^{(0)} = \mathbf{0} & \text{ on }\Gamma_D \times (0,T) \\
    \partial_t u^{(0)} = u^{(0)} = \mathbf{0} &  \text{ on } \Omega \times \{t=0\}
    \end{array}
    \right .
\end{equation*}
The standard discretization scheme is defined by a $\beta$-Newmark implicit finite differences scheme for the time domain and FEM for the space domain (using FEniCS environment). In this case, the time discretization is obtained in the usual two steps:
\begin{align*}
    \text{(Stage 1)} &\longrightarrow a_{n+1} = \frac{1}{\beta (\Delta t)^2} (u_{n+1}-u_{n}-(\Delta t)v_n) - \frac{1-2\beta}{2\beta}a_n\\
    \text{(Stage 2)}& \longrightarrow v_{n+1} = v_n + \Delta t((1-\gamma)a_n + \gamma a_{n+1})
\end{align*}
So that, the update scheme obtained at times $t_n = n+1$ in the form:
\begin{align*}
    \frac{\rho^{hom}}{\beta (\Delta t)^2} u_{n+1} - \nabla \cdot \sigma_C^{hom}( u_{n+1})  & - \frac{\gamma \Delta t}{\beta (\Delta t)^2} \nabla \cdot \sigma_D^{hom}(u_{n+1}) \\
    &= \frac{\rho^{hom}}{\beta (\Delta t)^2} (u_n + (\Delta t)v_n) + \rho^{hom}\frac{ (1-2\beta)}{2\beta} a_n \\
    & + \nabla \cdot \sigma_D^{hom}(v_n) + (1-\gamma)(\Delta t) \nabla\cdot \sigma_D^{hom}(a_n) \\
    & - \frac{(\Delta t)\gamma}{\beta (\Delta t)^2}\nabla \cdot \sigma_D^{hom}(u_n + (\Delta t)v_n) \\
    & - (\Delta t)\gamma\frac{1-2\beta}{2 \beta} \nabla \cdot \sigma_D^{hom}(a_n)
\end{align*}

\section{Frequency Domain Formulation}
The study on frequency domain relates to various aspects, ranging from the study of the elastic operator and its eigen-frequencies, the proximity of the fourier implementation within the device recording procedure to the computational restrictions limiting the time-domain implementation. 

To this end, let us take Fourier transform to the time-dependent problem in such a way that we consider particular solutions at frequency $\omega \in \mathbb{R}$ in the form:
\begin{equation}
    \label{FreqAnsatz}
    u^{\epsilon}(\mathbf{x},t) = \hat{u}^{\epsilon}(\mathbf{x},\omega) e^{i \omega t}
\end{equation}
In this case, the homogenized PDE problem (\ref{HomPDE-Model}, with its initial and boundary conditions becomes:
\begin{equation}
    \label{VectorPDE-Freq}
    \left \{
    \begin{array}{cc}
        -\rho^{hom} \omega^2 \hat{u}^{(0)}(\mathbf{x}) - \text{div }\sigma^{hom} (\hat{u}^{(0)}(\mathbf{x}))=\mathbf{0}  & \text{ in }  \Omega \\
        \sigma^{hom} (\hat{u}^{(0)}(\mathbf{x})) = \mathbf{C}^{hom}(\mathbf{x}): \mathbf{e} (\hat{u}^{(0)}(\mathbf{x})) & \text{ in } \Omega  \\
        \hat{u}^{(0)}(\mathbf{x}) = \mathbf{0} & \text{ on } \Gamma_D\\
        \sigma^{hom} (\hat{u}^{(0)}(\mathbf{x}))\cdot n = \hat{\mathbf{F}}(\mathbf{x}) & \text{ on } \Gamma_N \\
    \end{array}
    \right .
\end{equation}
where it was omitted the frequency dependency on the solution $\hat{u}^{(0)}$ for easiness of exposure and used $\Hat{F}(\omega, \mathbf{x})$ as the Fourier transform at frequency $\omega$ of $F(t, \mathbf{x})$.
