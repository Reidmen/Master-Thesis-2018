\begin{intro}
In the context of mechanical behavior in medicine, there are abundant question to be answer, fluctuating from the mathematical modelling itself of different organs and physiological structures, simulation of its experimental procedures to the verification and prediction of different factor with clinical interest such as density, stiffness constants, etc. From the computational modelling point-of-view, biological tissues define complex multiscale problems since there are multiple physical phenomena acting at various scales. Moreover in the bast majority of clinical applications the material is viewed in its continuous setting, thus the importance to contribute with mathematical models and numerical procedures to characterize non-trivial properties arising from the microscopic structures that affect the overall behavior in terms such as compressibility, elasticity or viscosity.

Nowadays, noninvasive resonance techniques corresponds to a unique and important aspect in the biomedical research and applications, as they allow detailed insight of important bone parameters in an noninvasive non-radiative fashion.

This work is oriented mainly in the modelling and numerical implementation of a novel ultrasound experimental procedure, in which via homogenization theory techniques it is studied the bone material mechanical behavior by means of a multiscale framework that incorporates the characteristic microstructure associated to the cortical bone, the so-called mesoscale.
Over such a model is derived the so-called cell problems that incorporated the relevant non-linearity of the two-scale asymptotic approximation being used thus providing a methodology to verify the experimental results and validate under controlled setting the \textit{ex-vivo} results.

By using the two-scale homogenization framework, we obtain the macroscopic equations governing the composite bone material that incorporates the intrinsic non-linear microstructural behavior.
It gives us also an explicit step-wise algorithm that first computes the material homogeneous structucture and then describes the macroscopic \footnote{Depending on the literature being used, the macroscopic behavior is also expressed as effective or slow-variable behavior, since is the one that describes the nature defined typically on application fields as evolution problems, i.e. time dependent PDE's.} model underlying the initial setting, allowing us an explicit formulation thus implementation of it using a state-of-art library called \texttt{FEniCS} to compute the solutions of such kind of systems.

By modelling implementing numerically the ultrasound experimental procedure proposed from \textit{Minonzio, Foiret} \cite{Foiret2014}, and \cite{Foiret2014} its then obtained by simulations the so-called \textit{Lamb} wave curves and singular values ressembling the real data. In a similar fashion, the solution of the homogenized coefficients that model the effective mechanical behavior are comparable with the reference literature, providing a standpoint for future studies.

On the other hand, from a theoretical stand-point its justified the two-scale asymptotic expansion being used by means of a convergence result. It relates the solution associated to the multiscale composite model of the experimental procedure to the effective bone model.

Similarly, its studied variations to the model problems to account effects of resonance behavior in the frequency domain, and instabilities observed in the spectral decomposition of the operator. 
Finally, it's studied the effects of viscoelasticity behavior in the model using relevant literature, searching for relevant viscous effects shown in literature and modelled by a small set of parameters modelled in an adequate mechanical behavior.


\end{intro}