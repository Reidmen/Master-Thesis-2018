\begin{conclusion}
%Things to mention:
%* Overview of the current literature work on biomechanical studies
%* What was done in the current investigation.

%* What was added with the investigation. 
%** FEM prediction of cell problems comparing them with PG results
%** Validation of MINONZIO procedure under 2-dimensional models, and study of 3-dimensional model studying the curvature and irregularity effects.
%** Proposed quality-factors under the homogenization two-scale convergence.

%* What is missing, necessity of greater samples to validate the models. 

%* What are the possible paths to follow or how to expands the current work.

The current advance of QUS devices, capable to record multi-array signal opens-up the study of new techniques to recover clinical-relevant data from which various material properties can be extracted. Nonetheless, it requires the validation and case-study of possible limitations of such devices in wide variety of setting to evaluate fidelity and therefore, assure clinical usage.

This works fills that gap, recreating under simplified configurations the modelling and generation of realistic wave-front recording that reproduce \textit{Lamb}-curves describing the material behavior.
By applying two microstructure types regarding the homogenization assumptions, it is obtained by FEM homogenized tensors that resemble PG predictions with $< 1 \%$ of RMSE for axial-related coefficients, while $> 10$ errors on fully anti-axial coefficients at high porosity values. Such error remains similar at changes from square to hexagonal and cubic type geometries, attributed then to the axial 2-dimensional assumptions of the characteristic microstructure, not adding possible anti-axial effects.

Modeling, discretization and simulations regarding the novel QUS method are done under 2-dimensional settings, both in time and frequency domains. Validation of the inverse problem predictions are done using test sets, thus ensuring correct \textit{Lamb}-modes identification from the homogenized materials simulated under specific model cases. Similarly, the numerical procedure is generalized to 3-dimensions under regular and irregular domain characterizing cortical bone on increasing complexity and simulations are done taking into account the computational and time resources necessary to generate sample data. In this sense, the high cost of the computational simulations to retrieve irregular models with enough fidelity become a bottleneck, therefore non-physically inspired model could be taken into account such as learning methods in a cost-functional perspective.
On the other hand, from a device-linked perspective, the time-domain model shows dependence from the directional measurement and therefore the sensibility and lack of robustness toward small changes in directional recordings. Such aspect must be considered in future device development and modeling, since it describes a fundamental defect that can negatively contribute towards an extensive device usage in standard clinical procedures.
\\
\newpage
The presence of viscous-like behavior on experimentally recorded signals becomes an relevant investigation factor, that could echo the processed signal. Several literature have tried to retrieve such contribution to the overall mechanical behavior, nevertheless the difficulty to separate such contribution from the other factors such as domain irregularities have become a challenge from the experimental setting as from the modelling perspective. Under such context, quality-factor are defined from a homogenized \textit{Kelvin-Voigt} viscoelastic model. It represent a first approach, formally validated, that recreates experimentally observed values under specific parameters and moreover, recovers the elastic coefficient values. 
Nevertheless, the lack of experimental data lets us conclude partial prediction of quality-factors only at high density values of cortical bone. 
\\

Directions to follow the current work are many, but investigations to tackle the numerical model under efficient methods, and study closely the effects of particular mesh irregularities are relevant to assess validation of the experimental procedure in more complex domains. Similarly, studies regarding the device sensibility can be taken into account, such as variation of directional measurements and moreover effects of device noise over the general setting. 
In a similar fashion, more complex viscoelastic models fitting experimental data should be considered since bone-skin and bone-marrow interactions contribute as damping effects on the recorded signal.
\end{conclusion}
