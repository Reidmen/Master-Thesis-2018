\chapter{Clinical Measurements}
Cortical Bone is a highly organized, hard and lightweight tissue representing approximately 80\% of the skeletal mass in an human adult defined by an clear hierarchy of micro-structures. From a functional description, scaling down to nanometers its observed important structures for the cortical tissue such as haversian systems, osteons, lamellae, collagen fibres, fibrils, and elementary mineral constituents with water \cite{Parnell2008}.

The standard description associated to a mechanical point of view is defining it as a two-phase composite material: a soft phase mainly of pores, containing organic fluid and soft tissues as cells, blood vessels, and a complex hard matrix phase with nerves distributed inside, being the matrix consisting mainly of hidroxyapatite and collagen. Porosity is distributed over various length scales, nevertheless only the two largest pores types: resorption cavities (with size of approx. $50-200 [\mu m]$ and haversian canal (size of approx. $50 [ \mu m ]$ contribute to the so-called mesoscale structure thus the mechanical behavior of the cortical bone itself, hence the mesoscale porosity which is characteristic of the higher level of organization in bone. 

\begin{figure}[!h]
	\centering
	\includegraphics[scale=.5]{images/ImgExt/246-2010_rec0964.pdf}
	\caption{Real sample of $\mu$-CT image obtained, slice 246 of stack. \textit{France, 2010}.}
	\label{muCT-Image}
\end{figure}

The study of such tissue gives us insights of various clinical parameters necessary to diagnose and reflect the status and quality of the bone itself, therefore a predict risk factors and possible problem affecting a wide range of patients. Under such setting, modelling and computational simulations of biological tissues enables us to go further and test new procedures, process and validate under specific enviroments the possible outputs becomes an important tool between the scientific study, and validation of new technologies and application devices in this area.

\section{Time-domain Modelling}

Sophisticated quantitative ultrasound (QUS) approaches under study \cite{Foiret2014} \cite{Minonzio2018}, are based on the axial transmission measurements which consist of guided waves recordings that propagate into and through the cortex in response to an ultrasonic excitation produced at the surface and then studying their response in the form of dispersion curves \footnote{From a physical perspective, its represented as the variation of wave number $k = 2 \pi f/c(f)$ as function of the frequency $f \in [0, 2]$ being $c(f)$ the phase velocity of the mode.}, i.e., by means of the \textit{Lamb}-waves nonlinear equations \cite{Rhee2007},
Waveguide characteristics such as thickness and porosity can then be deduced from the dispersion curves by finding the best fitting of theoretical waveguide model to experimental data after a signal processing step. Such procedure define in particular the inverse problem under consideration.

In this section, its described the experimental procedure proposed by \cite{Minonzio2018} used to study two bone mechanical properties by means of a ultrasound transducer.
It is also given a brief explanation of the setting involved in the clinical procedure and the assumptions being done to model the recorded signal then processed by a spectrum technique. 

\subsection{Experimental Procedure}
The bone sample studies are subjected to the transmission of the wave-guide, i.e., a wave propagated from the external surface generated by a transducer device. 
Explicitly such force can be considered in the form:
\begin{equation*}
    \mathbf{F}(\mathbf{x},t) = A e^{-\frac{(t-t_0)^2}{2\sigma_0^2}} cos(2 \pi \tau_0 (t-t_0)) \text{ on } \Gamma_N
\end{equation*}
where $A > 0$ denotes an amplitude, $t_0 > 0$ a central time, $\tau_0$ some period and $\Gamma_N$ surface boundaries where the force is applied.
Even thought the shape of long segments of cortical bone is not uniform with respect to thickness nor in its surface, because of the device shape and size, it can be considered that such local spatial variations in the geometry are minimum and moreover can be neglected \cite{Foiret2014}. \\

As the transducer device captures the wave-guide over the long axis of bone, the propagation effects given on the axial (or anti-plane) of the wave does not add more relevant features in such a way that the natural 3-dimensional cylindrical-like shape of the bone can be simplified without affecting the overall behavior of interest to a 2-dimensional plate shape domain modelling the coronal cut plane of propagation. \\
%%%   
%%%
%%% CHECKED TO THIS POINT!
%%%   
%%%
For such a 2-dimensional wave-guide model, the propagation is studied by means of the homogenization technique in elastic and viscoelastic behaviors. Such a homogenization procedure results necessary since the size of microstructure generates restrictive computational costs, and moreover the ability of such theory to model the microstructure observed for example in (\ref{muCT-Image}), i.e., containing the porosity level implicitly in the elastic coefficients obtained by means of two-scale asymptotic framework.


\begin{figure}[!h]
	\centering
	\includegraphics[width=0.7\textwidth]{images/ImgExt/SchematicPropagation.png}
	\caption{Schematic Procedure: from the experimental device to the homogenize idealization of the cortical bone.}
	\label{SchematicProp&Hom}
\end{figure} 

The important aspect for the transducer is the linear element array which contains several emitters and receiver. We model the experimental setting of the transducer applied on a patient as a elastodynamic model defined on the cortical bone, where we neglect the effects at the interface defined by the skin. 

In the case of simulations the number of emitters was fixed to 8 initially, but by the translation invariance of the elastic wave, the simulation only needed the emission using one emitter (or source), while the receptor where modeled by points in the surface containing the displacement vector.

\subsection{Signal Processing}
Each cycle of measurements consists of a sequential excitation of the $N^E > 0$ number of emitters located at boundaries $\Gamma_{e}$ as will be formalized later, which yield $N^E \times N^R$ time series arrays at each time of measurement.
Such a multi-channel series denoted by $\{ S(t_n, e_m,x_p) \}_{(n,p) \in [N^T]\times [N^R]}\}$ at fixed emitter $x_p, \, p \in \{1, \dots, N^E\}$ is then applied a 2-dimensional fast \textit{Fourier} transform obtaining $\{ \hat{S}(f_{\tilde{n}},e_m,k_{\tilde{p}}) \}_{(\tilde{n},\tilde{p}) \in [N^F]\times [N^K]}$\footnote{Which we identify following the literature of mechanics and ultrasound analysis as frequency and wavenumber respectively.}

Such Fourier transformed array $\hat{S}(\cdot, e_m, \cdot)$ for each emitter is then decomposed by singular values method (SVD), in the form:
\begin{equation*}
    P(e_m) \hat{D} P(e_m)^T = \hat{S}(e_m) \quad m \in \{1, \dots, N^E \}
\end{equation*}
which decompose the spectral signal in its main singular values thus describing the relevant modes of mechanical behavior.
So, let us define the following function on the first $N^E_*$ resonant modes by: 
\begin{equation*}
    L(P) := \sum \limits_{m = 1}^{N^E_*} P(e_m) \overline{P(e_m)}
\end{equation*}
Such function describes a measurement of the associated to the first $N^E_*$ recorded modes in a norm-like sense, thus describing the so-called \textit{Lamb} waves (or branches) of propagation \footnote{Several studies consider the usage of this kind of curves, as ones mainly used to describe the material destruction and mechanical behavior, and moreover in a non-invasive form.} of the material \cite{Rhee2007}. Experimentally, figure (\ref{RealLM-Image}) shows two particular cases of \textit{Lamb}-waves recordings, each describing the symmetric and anti-symmetric modes characteristic of each material.

\begin{figure}[!h]
	\centering
	\includegraphics[width=0.9\textwidth]{images/ImgExt/LW-RealCase.png}
	\caption{On the left side, $L(p)$ real-data plot describing \textit{Lamb}-modes for a case of $1.3 [mm]$ thickness plate whereas on the right side symmetric and anti-symmetric (red and blue respectively) \textit{Lamb}-modes (dashed line) with the experimental points obtained alongside the $L(p)$ functional curves calculate optimized to fit the experimental data characterizing the particular (porosity, thickness) pair \cite{Foiret2014}.}
	\label{RealLM-Image}
\end{figure}


The inverse problem associated is then formulated as
\begin{equation*}
    (Th.^*, Po.^*) = \underset{(Th., Po.) \in \mathcal{A}}{\text{argmax}} \int\limits_{f_{min}}^{f_{max}} \frac{1}{M} \sum_{m=1}^M \left \vert \langle \hat{D}(k_m, f_p)\, \vert e^{ik_m(Th., Po., f_p} \rangle \right \vert
\end{equation*}
being $M>0$ the number of modes under to consider. Thus, the formulation is re-expressed as the recuperation of $(Th.^*, Po.^*)$, two relevant biomedical parameters describing the quality of cortical bone.

An important aspect of such \textit{Lamb} curves is the fact that implicitly contain the overall mechanical behavior of the elastodynamical model, so that define a useful tool for validation and comparison of numerical simulation with respect to real data. 


Over such a domain we consider their behaviour of linear elastodynamic type, defined by a displacement $u(\mathbf{x},t)$ with constitutive equation of linear (following \textit{Hookes} law), with forces $\mathbf{F}(t)$ applied at a section of the surface domain.

\section{Main assumptions and Multiscale Modelling}

Computational modelling of biological tissues (in particular of bone) is a complex multiscale problem, since there are multiple physical phenomena interacting at various scales. Since in a bast majority of clinical applications the material are viewed as continuum, it's important to contribute in a mathematical model and numerical procedure to characterize non-trivial properties arising from the microscopic physical and geometrical structures affecting the overall behavior, in terms of effective compressibility (or viscosity) or elasticity. 

Formally we consider the following:
Let a bounded domain $\Omega \subset \mathbb{R}^d$ ($d = 2,3$) represent the composed material under study, modeled by an incompressible, elastic matrix and several inclusion defining the so-called mesoscale being described mechanically by an elastic of cylindrical shape periodically distributed.
For the sake of the experimental setting under consideration, the exterior boundary $\partial \Omega$ is decomposed as
\begin{equation*}
	\partial \Omega = \Gamma_D \dot\cup \Gamma_N
\end{equation*}
denoting $\Gamma_D, \Gamma_N$ the Dirichlet and Neumann part of the exterior boundary respectively.

Furthermore, let the small parameter $0 < \epsilon \ll 1$ denote the aspect ratio between the macroscopic variable $\mathbf{x} \in \Omega$ and the microscopic one $\mathbf{y}$ in the form: $\mathbf{y} = \frac{\mathbf{x}}{\epsilon}$ being $\mathbf{y} \in \mathbf{Y}$ and $\mathbf{Y} = (0,1)^d$ the cell structure. In this configuration, the unitary cell $\mathbf{Y}$, can be decomposed as
\begin{equation*}
	\mathbf{Y} = \mathbf{Y}_m \cup \Gamma \cup \mathbf{Y}_p 
\end{equation*}
where $\mathbf{Y}_m$ and $\mathbf{Y}_p$ are defined as the domains occupied by the matrix and porous part respectively, and $\Gamma$ denotes the interface between them\footnote{This model is inspired in the development of the Homogenization of the elastic operator in a porous media proposed on \cite{christensen1982theory}. Nevertheless, this kind of configurations are typical in the two-scale homogenization literature \cite{panasenko2005multi-scale}, \cite{Boughammoura2013} to mention updated references.}.

Such a cell is assumed to be periodically distributed along the material, defining its highly oscillatory structure, suitable to use the the Homogenization framework. We consider the composite material of study with elastic properties having oscillation rate $\epsilon$, represented in the domain decomposition:
\begin{equation*}
	\overline{\Omega} = (\overline{\Omega}\setminus \Omega_1^{\epsilon}) \cup \overline{\Omega}^{\epsilon}_1
\end{equation*}
where we have defined
\begin{equation*}
    \overline{\Omega}^{\epsilon}_1 = \bigcup_{\mathbf{x} \in \mathbf{T}_{\epsilon}} \epsilon ( \mathbf{x} + \mathbf{Y} )
\end{equation*}
being the tessellation $\mathbf{T}_{\epsilon}$ the subset in $\mathbb{Z}^d$ of all points satisfying the conditions
\begin{equation*}
    \epsilon (\mathbf{x} + \mathbf{Y}) \subset \Omega, \quad \rho(\epsilon(\mathbf{x}+\mathbf{Y}), \partial \Omega) \geq \epsilon
\end{equation*}

Let us note in particular that for fixed $\epsilon >0$ and all $\mathbf{x} \in \Omega$ we have a unique decomposition $\mathbf{x}/\epsilon = \mathbf{x}_{\mathbf{T}_{\epsilon}(\mathbf{x})} + \mathbf{y}$ where $\mathbf{x}_{\mathbf{T}_{\epsilon}(\mathbf{x})}$ denotes the element in $\Omega$ of the tessellation.

The above considerations let us define finally the material coefficients as second order rank tensors for component $i,j,k,l \in \{1,\dots, d\}$ in the form:
\begin{equation*}
    C_{ijll}(\frac{\mathbf{x}}{\epsilon}) = C_{ijkl}(\mathbf{y}) \quad \text{for } \frac{\mathbf{x}}{\epsilon} = \mathbf{x}_{\mathbf{T}_{\epsilon}(\mathbf{x})} + \mathbf{y}, \, \mathbf{x} \in \Omega_1^{\epsilon}
\end{equation*}
and being $C_{ijkl}(\frac{\mathbf{x}}{\epsilon}) = 0$ if $\mathbf{x} \in \overline{\Omega} \setminus \Omega_1^{\epsilon}$.

\begin{figure}[!h]
	\centering
	\includegraphics[width=0.8\textwidth]{images/HomSchemes/HomBasicScheme.pdf}
	\caption{Homogenization Scheme of Bone. A composite periodic material at the long direction.}
	\label{HomBasicScheme}
\end{figure}

\begin{rem}
We model the material confined within the inclusion as a elastic (emulating a mechanical behavior of blood mixture mainly made of saturated static fluid) with volume $\epsilon^d \vert \mathbf{Y}_p \vert$ embedded in a elastic material (modeled mainly of hydroxipatite and collagen) filling a unit cell of volume $\epsilon^d \vert \mathbf{Y} \vert$.
We define, after scaling, the porosity associated to the material as the fraction of volume occupied by the gas in the unitary cell, i.e., 
\begin{equation*}
\phi = \frac{\vert \mathbf{Y}_p \vert}{\vert \mathbf{Y} \vert} = \vert \mathbf{Y}_p \vert
\end{equation*}
which will define the material coefficients dependent of such parameters.
\end{rem}

We denote the displacement $u(\mathbf{x},t) \in \mathbf{H}^1(\Omega)$ with behaviour of linear elastic type subjected to surface forces $\mathbf{F}(t) \in L^2 (0, T)$, i.e., the material obeys the following initial boundary value problem:

\begin{equation*}
    \left \{
    \begin{aligned}
        \rho (\mathbf{x}) \partial_{tt} u(\mathbf{x},t) - \nabla \cdot \sigma (u) & = \mathbf{0}, \text{ in } \Omega \times (0, T) \\
        \sigma(u(\mathbf{x},t))_{ij} = \mathbf{C}_{ijkl} \mathbf{e}_{kl}(u(\mathbf{x},t)) & \text{ in } \Omega \times (0, T) \\
        u(\mathbf{x},t) = \mathbf{0} & \text{ on } \Gamma_D \times (0, T) \\
    \sigma(u(\mathbf{x},t)) \cdot n = \mathbf{F}(t) & \text{ on } \Gamma_N \times (0,T)
    \end{aligned}
    \right .
\end{equation*}

where $\mathbf{C}(\mathbf{x})$ denotes the elasticity tensor and $\mathbf{e}(u(\mathbf{x},t)) = \frac{1}{2}\big( \nabla u(\mathbf{x},t) + \nabla u(\mathbf{x},t)^{T}\big)$ denotes the symmetric gradient.


