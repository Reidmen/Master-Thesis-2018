\begin{intro}
In the context of mechanical behavior in medicine, there are abundant question to answer, fluctuating from the mathematical modelling itself of different tissues, physiological structures, simulation of its experimental procedures to the verification and prediction of different factor of clinical interest such as density, stiffness constants, etc. From a computational modelling point-of-view, biological tissues define complex multiscale problems since there are multiple physical phenomena acting at various scales. Moreover, from the bast majority of clinical applications the materials are viewed in its continuous setting, thus the importance to contribute with mathematical models and numerical procedures to characterize non-trivial properties arising from the microscopic structures that affect the overall behavior in terms such as compressibility, elasticity or viscosity.
One of particular interest is the human bone, an absorbing complex composite structure that have an irregular hollow-like interior filled with marrow, surrounded by soft tissue and muscles. Over which it's defined the so-called bone quality, a composite of properties that enables bone to resist fracture. In this sense, when more bone is removed than added in the same time span, the so-called \textit{osteoporosis} condition appear, a established and well-defined disease from the World Health Organization, affecting more than 75 million of people in Europe, United States and Japan combined. being the major cause of fractures

Nowadays, noninvasive ultrasound techniques corresponds to a unique and important aspect in the biomedical research and applications, as they allow detailed insight of important bone parameters with a lower-cost, non-invasive and non-radiative properties that expects to challenge the gold-standard techniques used in clinical procedures. Under this setting, \textit{Minonzio et. al.} \cite{Minonzio2018} proposed a technique of wave-guide propagation to recover two relevant parameters that describe bone quality, the cortical and thickness of cortical bone, using an inverse-like problem formulation. Several questions must be addressed to successfully validate the technique, ranging from the study such as the underlying theory used, the effect of domain irregularities, robustness as well as the validation under controlled scenarios.

This work is oriented mainly in the modelling and numerical implementation of such novel ultrasound experimental procedure, in which via homogenization theory techniques it's studied the bone material mechanical behavior and their description using the \textit{Lamb}-curves theory, in which by means of a multiscale framework it incorporates the characteristic microstructure associated to the cortical bone, the so-called mesoscale. 
Over such a model is derived the cell problems that contains the relevant non-linearity of the two-scale asymptotic approximation being used, thus providing a methodology to verify and validate using simulated data and with respect to \textit{ex-vivo} results under controlled settings.

By using the two-scale homogenization framework, it's studied the macroscopic equations governing the composite bone material that incorporates the intrinsic non-linear microstructural behavior.
It gives us also an explicit step-wise algorithm that first computes the material homogeneous structucture and then describes the macroscopic \footnote{Depending on the literature being used, the macroscopic behavior is also expressed as effective or slow-variable behavior, since is the one that describes the nature defined typically on application fields as evolution problems, i.e. time dependent partial differential equation system.} model underlying the initial setting, allowing us an explicit formulation, thus implementation using the state-of-art \texttt{FEniCS} \cite{logg2012automated} library to compute the solutions of such kind of systems.

By implementing the ultrasound experimental procedure proposed from \textit{Minonzio, Foiret} \cite{Foiret2014}, \cite{Minonzio2018} its simulated  \textit{Lamb}-mode curves and singular values resembling the real behavior, assessing the parameter dependency and fidelity. In this context, it's studied properties of the elastic operator to account effects of resonance behavior in the frequency domain and instabilities observed in their spectral decomposition.
Similarly, numerically obtained homogenized elastic coefficients that model the effective mechanical behavior are compared with the reference literate, providing a standpoint towards further modification regarding symmetry, and domain configuration.
Finally, from a theoretical stand-point its studied the justification of the two-scale asymptotic expansion concluding on a convergence result, relating the multiscale solution of the composite material and the homogenized solution. Moreover the extension toward a viscoelasticity formulation regarding damping effects naturally appearing from the complex interaction between several tissues on the forearm.

\end{intro}